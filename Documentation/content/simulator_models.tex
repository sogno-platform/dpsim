\chapter{Simulator Models}

\section{Basic Components}

\subsection{Inductor}

\begin{equation}
        \frac{di(t)}{dt} = \frac{1}{L} \cdot v(t) - j \omega \cdot i(t)
\end{equation}

Apply trapezoidal rule:

\begin{equation}
        i(t) = i(t - \Delta t) + \frac{\Delta t}{2} \left[ \frac{1}{L} \cdot v(t) - j \omega \cdot i(t) + \frac{1}{L} \cdot v(t - \Delta t) - j \omega \cdot i(t - \Delta t) \right]
\end{equation}

\begin{align}
        a &= \frac{\Delta t}{2L} \\
        b &= \frac{\Delta t \omega}{2}
\end{align}

\begin{align}
        i(t) &= i(t - \Delta t) + a \cdot v(t) - j b \cdot i(t) + a \cdot v(t - \Delta t) - j b \cdot i(t - \Delta t) \\
        i(t) &= \frac{1-b^2-j2b}{1+b^2} \cdot i(t - \Delta t) + \frac{a-jab}{1+b^2} \left[ v(t) - v(t - \Delta t) \right]
\end{align}

\subsection{Capacitor}

\begin{equation}
        \frac{dv(t)}{dt} = \frac{1}{C} \cdot i(t) - j \omega \cdot v(t)
\end{equation}

Apply trapezoidal rule:

\begin{equation}
        v(t) = v(t - \Delta t) + \frac{\Delta t}{2} \left[ \frac{1}{C} \cdot i(t) - j \omega \cdot v(t) + \frac{1}{C} \cdot i(t - \Delta t) - j \omega \cdot v(t - \Delta t) \right]
\end{equation}

\begin{align}
        a &= \frac{\Delta t}{2C} \\
        b &= \frac{\Delta t \omega}{2}
\end{align}

\begin{align}
        v(t) &= v(t - \Delta t) + a \cdot i(t) - j b \cdot v(t) + a \cdot i(t - \Delta t) - j b \cdot v(t - \Delta t) \\
        i(t) &= -i(t- \Delta t) + \frac{1+jb}{a} \cdot v(t) + \frac{-1+jb}{a} \cdot v(t - \Delta t)
\end{align}

\section{Synchronous Machine}

The model is according to \cite{wang2010methods} and \cite{kundur1994power}. 

\subsubsection{Prerequisites}
Park's transformation is commonly used to achieve a model with static parameters:
%
\begin{equation}
\mathbf{K_s} = \frac{2}{3}
 \begin{bmatrix} 
  \cos \theta & \cos(\theta-\frac{2\pi}{3}) & \cos(\theta+\frac{2\pi}{3}) \\
  \sin \theta & \sin(\theta-\frac{2\pi}{3}) & \sin(\theta+\frac{2\pi}{3}) \\
  \frac{1}{2} & \frac{1}{2} & \frac{1}{2}
 \end{bmatrix}
\end{equation}
%
Note that the scaling factor $\frac{2}{3}$ is not always used. 

\subsubsection{Model}

The mechanical equations are:
%
\begin{align}
\frac{d\theta_r}{dt} &= \omega_r \\
\frac{d\omega_r}{dt} &= \frac{P}{2J} (T_e-T_m)
\end{align}
%
where $\theta_r$ is the rotor position, $\omega_r$ is the angular electrical speed, $P$ is the number of poles, $J$ is the moment of inertia, $T_m$ and $T_e$ are the mechanical and electrical torque, respectively. Motor convention is used for all models. 

The electrical model in the phase domain is described by the following equations:
%
\begin{align}
  \mathbf{v}_{abcs} &= \mathbf{R}_s \mathbf{i}_{abcs} + \frac{d}{dt} \boldsymbol{\lambda}_{abcs} \\
  \mathbf{v}_{qdr} &= \mathbf{R}_r \mathbf{i}_{qdr} + \frac{d}{dt}  \boldsymbol{\lambda}_{qdr}
\end{align}
%